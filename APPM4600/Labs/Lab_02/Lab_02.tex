\documentclass{article}
\usepackage[utf8]{inputenc}
\usepackage[margin=1in,left=1.5in,includefoot]{geometry}
\usepackage{booktabs}
\usepackage{graphicx}

%%%%%%%%%%%%%%% Header & Footer Stuff %%%%%%%%%%%%%%%
\usepackage{fancyhdr}
\pagestyle{fancy}
\lhead{Magnus Miller}
\rhead{MATH 4600} 
\chead{\textbf{Lab 02}}
\lfoot{\fontsize{10pt}{12pt}\selectfont Dept. of Applied Mathematics}
\rfoot{\fontsize{10pt}{12pt}\selectfont University of Colorado Boulder}
\cfoot{\fontsize{10pt}{12pt}\selectfont Page \thepage}

%%%%%%%%%%%%%%% The Main Document %%%%%%%%%%%%%%%
\begin{document}

\begin{center}
 \LARGE\bfseries Lab \# 2
\end{center}
\begin{center}
    ~01/21/2025~
\end{center}
 \line(1,0){430}

%%%%%%%%%% QUESTIONS %%%%%%%%%%
\section{Introduction}
In this lab, I became acquainted with Github and its usage in the command line. I had previously created my Github account (\textit{@magnuswmiller}) which I used to create my repository for this course named \textit{APPM4600\_Miller}. After creating and cloning my repository, I practiced pulling, making commits, pushing, and working in the command line. I also shared the repository with the account \textit{APPM4600instruct}.

\section{Process \& Experience}
In previous work, I have had extensive experience using \textit{Git} and \textit{GitHub} whether it be for academic work, elective clubs, and even personal work. Due to this, the experience of creating \textbf{SSH} keys, connecting it to my account, creating and cloning a repository, and writing my \textbf{README.md} file was quite easy. Similarly, the process of pulling, committing, and pushing work between the local and remote repositories was simple.

\section{Further Readings}
After completing the deliverables portion of the lab, I continued on to read \textbf{Section 6} as well as \textbf{Section 7A-E} regarding branching within repositories. Some of the information introduced in these sections was review from previous experiences while the later sections were new. I haven't needed to deal with merging of merge conflicts yet so this information was quite interesting.

\end{document}