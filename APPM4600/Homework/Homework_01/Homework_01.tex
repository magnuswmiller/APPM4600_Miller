\documentclass{article}
\usepackage[utf8]{inputenc}
\usepackage[margin=1in,left=1.5in,includefoot]{geometry}
\usepackage{booktabs}
\usepackage{graphicx}
\usepackage{amsmath}

%%%%%%%%%%%%%%% Header & Footer Stuff %%%%%%%%%%%%%%%
\usepackage{fancyhdr}
\pagestyle{fancy}
\lhead{Magnus Miller}
\rhead{MATH 4600} 
\chead{\textbf{Homwork 01}}
\lfoot{\fontsize{10pt}{12pt}\selectfont Dept. of Applied Mathematics}
\rfoot{\fontsize{10pt}{12pt}\selectfont University of Colorado Boulder}
\cfoot{\fontsize{10pt}{12pt}\selectfont Page \thepage}

%%%%%%%%%%%%%%% The Main Document %%%%%%%%%%%%%%%
\begin{document}

\begin{center}
 \LARGE\bfseries HOMEWORK \# 01
\end{center}
\begin{center}
    01/24/2025
\end{center}
 \line(1,0){430}

%%%%%%%%%% QUESTIONS %%%%%%%%%%
\section{Question 1}
Problem 1 revolves around the following polynomial found in \ref{Eqn_1}. In addition to the equation in \ref{Eqn_1}, we are given an equivalent expression which can be found in \ref{Eqn_2}. The following parts of this question ask to plot \(p(x)\) in different ways and to compare the results.
\begin{equation}
    \label{Eqn_1}
    p(x) = (x-2)^9
\end{equation}
\begin{equation}
    \label{Eqn_2}
    p(x) = x^9 - 18x^8 + 144x^7 - 672x^6 + 2016x^5 - 4032x^4 + 5376x^3 - 4608x^2 + 2304x - 512
\end{equation}

\subsection{Plotting using the coefficients}
Part I asks us to plot \(p(x)\) using the coefficient form of the polynomial found in \ref{Eqn_2} over the interval \([1.920 : 2.080]\). This was accomplished using the \textbf{numpy} library and it's \textit{arange()} function in coordination with the \textbf{matplotlib.pyplot} library. The following figure, \ref{fig:p_coef}, was created.
\begin{figure}[h!]
    \centering
    \includegraphics[width=0.5\linewidth]{Homework_01/p_coef.png}
    \caption{Plotting of \(p(x)\) using the expression in \ref{Eqn_2}.}
    \label{fig:p_coef}
\end{figure}

\subsection{Plotting using the expression}
Part II also asks us to plot \(p(x)\) using the expression form of the polynomial found in \ref{Eqn_1}. As in Part I, the \textbf{numpy} and \textbf{matplotlib} libraries were used. The following figure, \ref{fig:p_expression} was produced.
\begin{figure}[h!]
    \centering
    \includegraphics[width=0.5\linewidth]{Homework_01/p_expression.png}
    \caption{Plotting of \(p(x)\) using the expression in \ref{Eqn_1}.}
    \label{fig:p_expression}
\end{figure}
\newpage

\subsection{Comparison}
To better compare the two methods, an overlay of the plots was created using a similar method which can be seen below in \ref{fig:p_comparison}.
\begin{figure}[h!]
    \centering
    \includegraphics[width=0.5\linewidth]{Homework_01/p_comparison.png}
    \caption{Plotting of \(p(x)\) using both methods.}
    \label{fig:p_comparison}
\end{figure}
As seen in Figure \ref{fig:p_comparison}, the plot of \(p(x)\) using \ref{Eqn_1} is far smoother than the plot using \ref{Eqn_2}.
The discrepancy arises due to floating-point round-off errors in evaluating the polynomial via its coefficients. The expression \( (x-2)^9 \) is computationally simpler, involving fewer operations, and hence more accurate. The evaluation using coefficients suffers from catastrophic cancellation, especially near \(x=2\), where \( (x-2) \) approaches zero.


\section{Question 2}
Problem 2 is centered around calculations that are subject to cancellation issues. The problem offers three expressions, and we are tasked with performing the calculations as to avoid cancellation.

\subsection{Part I}
The calculation for Part I can be found below in \ref{Eqn_3}.
\begin{equation}
    \label{Eqn_3}
    \sqrt{x+1} - 1\mbox{, for x} \simeq \mbox{0}
\end{equation}
Considering the equation above, as x approaches 0, the expression approaches \(sqrt{1} -1\) which we know is equal to zero. To absolve this issue, one can multiply the expression by its conjugate as shown below in \ref{Eqn_4}.
\begin{equation}
    \label{Eqn_4}
    (\sqrt{x+1} - 1)*\frac{\sqrt{x+1}+1}{\sqrt{x+1}+1} = \frac{(\sqrt{x+1} - 1)(\sqrt{x+1}+1)}{\sqrt{x+1}+1} = \frac{x}{\sqrt{x+1}+1}
\end{equation}
As shown, the final expression includes no subtraction, therefore avoiding the issue of cancellation.

\subsection{Part II}
The calculation for Part II can be found below in \ref{Eqn_5}.
\begin{equation}
    \label{Eqn_5}
    \sin{x} - \sin{y} \mbox{, for x} \simeq \mbox{y}
\end{equation}
To avoid the cancellation in the expression above, one can use a trigonometric substitution using sum identities as seen below in \ref{Eqn_6}.
\begin{equation}
    \label{Eqn_6}
    \sin{x} - \sin{y} = 2\cos{(\frac{x+y}{2})}\sin{(\frac{x+y}{2})}
\end{equation}
As shown, the expression includes no subtraction. Since there is no subtraction, the expression avoids the issue of cancellation.

\subsection{Part III}
The calculation for Part III can be found below in \ref{Eqn_7}.
\begin{equation}
    \label{Eqn_7}
    \frac{1-\cos{x}}{\sin{x}} \mbox{, for x} \simeq \mbox{0}
\end{equation}
To avoid the cancellation in the expression above, one can use a trigonometric substitution using half-angle identities as seen below in \ref{Eqn_8}
\begin{equation}
    \label{Eqn_8}
    \frac{1-\cos{x}}{\sin{x}} = \frac{\sin{x}}{1+\cos{x}} = \tan{\frac{x}{2}}
\end{equation}
As shown, the expression includes no subtraction. Since there is no subtraction, the expression avoids the issue of cancellation.


\section{Question 3}
Question 3 revolves around the function below in \ref{Eqn_9}. The question asks to find the function's second degree Taylor polynomial to use as an approximation for \(f(0.5)\). The question then begins to inquire about the error in the approximation.
\begin{equation}
    \label{Eqn_9}
    f(x) = 1 + x + x^3)\cos{(x)}, \mbox{ about } x = 0
\end{equation}
Given the above equation, we can formulate the second-degree polynomial as shown below in \ref{Eqn_10}.
\begin{equation}
    \label{Eqn_10}
    P_2(x) = f(c) + f'(c)(x-c) + \frac{f''(c)}{2!}(x-c)^2
\end{equation}
Since it is centered around \(x=0\), \ref{Eqn_10} becomes the following.
\[
P_2(x) = f(0) + f'(0)(x) + \frac{f''(0)}{2!}(x)^2
\]
Let us now find the first and second derivative of the function \(f(x)\) about \(x=0\).
\[
f(x) = (1 + x + x^3)\cos{(x)}
\]
\[
f'(x) = \left(3x^{2} + 1\right) \cos\left(x\right) - \left(x^{3} + x + 1\right) \sin\left(x\right)
\]
\[
f''(x) = \left(-6x^{2} - 2\right) \sin\left(x\right) + \left(-x^{3} + 5x - 1\right) \cos\left(x\right)
\]
When the above expressions are evaluated at \(x=0\), we get the following.
\[
f(0) = 1
\]
\[
f'(0) = 1
\]
\[
f''(0) = -1
\]
With these values, the second-degree Taylor polynomial now becomes the following in \ref{Eqn_11}.
\begin{equation}
    \label{Eqn_11}
    P_2(x) = 1 + x - \frac{1}{2}x^2
\end{equation}
With the second-degree Taylor polynomial for \(f(x)\) centered about \(x=0\), we can continue to the following parts of the question.

\subsection{Approximation of \(f(0.5)\) and the error}
In Part I we will use \(P_2(x)\) to approximate \(f(0.5)\) as shown below in \ref{Eqn_12}.
\begin{equation}
    \label{Eqn_12}
    f(0.5) \approx P_2(0.5) = 1 + (0.5) - \frac{1}{2}(0.5)
\end{equation}
\[
f(0.5) \approx \frac{3}{2} - \frac{1}{4} = \frac{5}{4}
\]
With the approximation, we can now calculate the upper bound for the error of the approximation using the error formula below.
\[
|f^{(n+1)}(z)| \leq M \mbox{ where } z \in [a,x]
\]
\[
R_n(0.5) = \frac{M}{6}(0.5)^3
\]
Using this and \(a = 0\), \(x = 0.5\), and \(\left(x^{3} - 17x + 1\right) \sin\left(x\right) + \left(3 - 9x^{2}\right) \cos\left(x\right)\), we know \(M =  3\) so \(R_n(0.5) = \frac{1}{16}\). Using a calculator, we calculated the actual value of \(f(0.5)\) to be \(1.624938125\). Therefore the actual error, \(|f(0.5) - P_2(0.5)\), is \(0.374938125\). The actual error is far greater than the theoretical upper bound for the error.

\subsection{Generalizing the error bound}
Using most of the same formulae from Part I, we can generalize the error bound for \(P_2(x)\) as an approximation of \(f(x)\) is the remainder term of the taylor polynomial as follows in \ref{Eqn_13}.
\begin{equation}
    \label{Eqn_13}
    R_2(x) = \frac{M}{3!}(x)^3 \mbox{ where } M = max|f^{(3)}(z)| \mbox{ where } z \in [0,x]
\end{equation}
\subsection{Approximating the integral of \(f(x)\)}
Part III asks us to approximate the integral \(\int_{0}^{1}f(x)dx\) using the second-degree Taylor polynomial. That is, we want to approximate the integral using \(\int_{0}^{1}P_2(x)dx\) which can be seen below in \ref{Eqn_14}.
\begin{equation}
    \label{Eqn_14}
    \int_{0}^{1}P_2(x)dx = \int_{0}^{1} 1 + x - \frac{1}{2}x^2 dx
\end{equation}
\[
= \int_{0}^{1} 1 dx + \int_{0}^{1} x dx - \int_{0}^{1} \frac{1}{2}x^2 dx
\]
\[
= 1 + \frac{1}{2} - \frac{1}{6}
\]
\[
= \frac{4}{3}
\]
\subsection{Estimating the error of the approximation}
Using Part II, we can calculate the remainder term of the polynomial over the interval \(z \in [0,1]\) as seen below in \ref{Eqn_15}.
\begin{equation}
    \label{Eqn_15}
    R_2(x) = \frac{M}{3!}(x)^3 \mbox{ where } M = max|f^{(3)}(z)| \mbox{ where } z \in [0,1]
\end{equation}
\[
R_2(x) = \frac{15.8638786073}{6}(0.5)^3
\]
\[
R_2(x) = 0.330497471
\]


\section{Question 4}
Question 3 revolves around the quadratic equation and its roots when given the inputs \(a=1\), \(b=-56\), \(c=1\).
\subsection{}
Given the quadratic equation and the above inputs, we can assume that we can calculate the roots to three decimal places as shown below in \ref{Eqn_16}.
\begin{equation}
    \label{Eqn_16}
    r_{1,2} = \frac{-b\pm \sqrt{b^2-4ac}}{2a}
\end{equation}
\[
r_{1,2} = 28 \pm \sqrt{(-56^2)(4)}
\]
\[r_1 = 55.982 \mbox{ and } r_2 = 0.018\]
When using a calculator, we get the following values for the roots: \(r_1 = 55.98213716\) and \(r_2 = 0.01786284073\).
Now that we have calculated the two roots, we can now calculate the relative roots using as shown below in \ref{Eqn_17}.
\begin{equation}
    \label{Eqn_17}
    \mbox{Relative Error} = \frac{|r_{exact} - r_{computed}|}{r_{exact}}
\end{equation}
So calculating the relative error for each root:
\[
e_1 = 2.45006723\times10^{-6}
\]
\[
e_2 = 7.678469067\times10^{-6}
\]
\subsection{Better methods of approximating roots}
In this section, we consider the method that can calculate to three decimals that was used in Part I to be the "bad" method. We are also given the following in \ref{Eqn_18}.
\begin{equation}
    \label{Eqn_18}
    (x-r_1)(x-r_2) = 0
\end{equation}
Using \ref{Eqn_18}, we get the following.
\[
(x-r_1)(x-r_2) = x^2 - xr_1 - xr_2 + r_1r_2
\]
When we compare to the quadratic \(ax^2 + bx + c = 0\), we see that \(a = \frac{x^2}{x^2} = 1\), \(b = \frac{-x(r_1+r_2)}{x} = -(r_1+r_2)\), and \(c = r_1r_2\). This gives us the following relations.
\[
r_1 + r_2 = \frac{-b}{a}
\]
\[
r_1r_2 = \frac{c}{a}
\]
Despite being useful for checking work when finding roots, these relations do not help in finding the roots any more accurately as reconfiguring the equations to solve for either \(r_1\) or \(r_2\) brings you back to the quadratic equation which we have assumed to only be solvable to three decimal places.


\section{Question 5}

Question 5 revolves around the idea of cancellation of terms. The question considers computing \(y = x_1 - x_2\) where \~{x}$_1$ \(= x_1 + \Delta x_1\) and \~{x}$_2$\( = x_2 + \Delta x_2\) are approximations to the exact values. It is important to note that if the operation is carried out exactly, we get \~{y}\( = y + (\Delta x_1 + \Delta x_2\) where \(\Delta y = \Delta x_1 + \Delta x_2\). To answer the following parts of this question, python was used.
\subsection{Absolute and relative errors}
Given that the absolute error is \(|\Delta y| = |\Delta x_1 - \Delta x_2|\), we know that the bound for the absolute value is \(|\Delta x_1| + |\Delta x_2|\). We also know that the relative error is given by \(\frac{|\Delta y|}{|y|}\). Since we know \(|\Delta y|\) is bounded by \(|\Delta x_1| + |\Delta x_2|\), the bound of the relative error depends on \(|y|\) in the denominator. That is, when \(x_1\) and \(x_2\) are very small, the relative error is very large and when they are very large, the relative error is very small. Using python, I set the smaller set of \(x\)'s to be \(0.005\) and \(0.000999999\) which gave a relative error value of \(4.9999987500003115\times10^{-5}\). When using the large values for \(x_1\) and \(x_2\), (\(1000000.0\) and \(9999999.0\), the relative error was \(2.22222246913583\times10^{-14}\).

\subsection{Avoiding subtraction}
In Part II, we are given the following expression in \ref{Eqn_19}.
\begin{equation}
    \label{Eqn_19}
    cos(x + \delta) - cos(x)
\end{equation}
To manipulate the expression into one without subtraction, we can rewrite the expression using a trigonometric sum identity as shown below.
\begin{equation}
    \label{Eqn_20}
    cos(x + \delta) - cos(x) = -2\sin{\frac{(x + \delta) + x}{2}\sin{\frac{(x + \delta) - x}{2}}}
\end{equation}
\[
= -2\sin{(\frac{2x + \delta}{2})\sin{(\frac{x}{2})}}
\]
Using \textbf{python} and the \textbf{numpy} and \textbf{matplotlib} libraries, the difference between the function in \ref{Eqn_19} and the function in \ref{Eqn_20} was plotted which can be seen below in Figures \ref{fig:Diff_pi} and \ref{}. It is important to note that two values of \(x\) were chosen: \(x=\pi\), and \(x=10^6\).
\begin{figure}[h!]
    \centering
    \includegraphics[width=0.5\linewidth]{Homework_01/Diff_pi.png}
    \caption{Difference of expressions using \(x=\pi\).}
    \label{fig:Diff_pi}
\end{figure}
\begin{figure}[h!]
    \centering
    \includegraphics[width=0.5\linewidth]{Homework_01/Diff_10^6.png}
    \caption{Difference of expressions using \(x=10^6\).}
    \label{fig:enter-label}
\end{figure}

\subsection{Taylor series and coming up with my own algorithm}
For Part III, I wasn’t entirely sure how to approach the problem. The goal was to use the Taylor expansion of \(\cos(x + \delta)\) to approximate \(\cos(x + \delta) - \cos(x)\). I started with the Taylor series:
\[
\cos(x + \delta) \approx \cos(x) - \delta \sin(x) - \frac{\delta^2}{2} \cos(x).
\]
From this, I could subtract \(\cos(x)\) to get:
\[
\cos(x + \delta) - \cos(x) \approx -\delta \sin(x) - \frac{\delta^2}{2} \cos(x).
\]
I wasn't the most sure in this path in terms of being able to fully answer the question let alone in being able to produce figures and interpret them accurately.


\end{document}

